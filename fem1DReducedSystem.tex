%----------------------------------------------------------------------------------------
%	PACKAGES AND THEMES
%----------------------------------------------------------------------------------------
\documentclass[aspectratio=169,xcolor=dvipsnames]{beamer}
\usetheme{SimplePlus}

\usepackage{hyperref}
\usepackage{graphicx} % Allows including images
\usepackage{booktabs} % Allows the use of \toprule, \midrule and \bottomrule in tables
\usepackage{ragged2e}
\usepackage{etoolbox}

\apptocmd{\frame}{}{\justifying}{}

%%----------------------------------------------------------------------------------------
%%	TITLE PAGE
%%----------------------------------------------------------------------------------------
%
%\title[short title]{Simple Beamer Theme} % The short title appears at the bottom of every slide, the full title is only on the title page
%\subtitle{Subtitle}
%
%\author[Pin-Yen] {Pin-Yen Huang}
%
%\institute[NTU] % Your institution as it will appear on the bottom of every slide, may be shorthand to save space
%{
%    Department of Computer Science and Information Engineering \\
%    National Taiwan University % Your institution for the title page
%}
%\date{\today} % Date, can be changed to a custom date
%
%
%%----------------------------------------------------------------------------------------
%%	PRESENTATION SLIDES
%%----------------------------------------------------------------------------------------

\begin{document}

\begin{frame}
  \begin{block}{}\justifying
    Per exemple, per \texttt{nDiv} $=5$ (5 elments) el sistema acoplat
    resulta,
  \begin{equation}\label{eq:coupledSystem}
    \begin{pmatrix}
      K_{1,1} & K_{1,2} & K_{1,3} & K_{1,4} & K_{1,5} & K_{1,6} \\
      K_{2,1} & K_{2,2} & K_{2,3} & K_{2,4} & K_{2,5} & K_{2,6} \\
      K_{3,1} & K_{3,2} & K_{3,3} & K_{3,4} & K_{3,5} & K_{3,6} \\
      K_{4,1} & K_{4,2} & K_{4,3} & K_{4,4} & K_{4,5} & K_{4,6} \\ 
      K_{5,1} & K_{5,2} & K_{5,3} & K_{5,4} & K_{5,5} & K_{5,6} \\ 
      K_{6,1} & K_{6,2} & K_{6,3} & K_{6,4} & K_{6,5} & K_{6,6}   
    \end{pmatrix}
    \begin{pmatrix}
      U_{1}\\
      U_{2}\\
      U_{3}\\
      U_{4}\\
      U_{5}\\
      U_{6}
    \end{pmatrix} =
    \begin{pmatrix}
      F_{1}\\
      F_{2}\\
      F_{3}\\
      F_{4}\\
      F_{5}\\
      F_{6}
    \end{pmatrix} +
    \begin{pmatrix}
      Q_{1}\\
      Q_{2}\\
      Q_{3}\\
      Q_{4}\\
      Q_{5}\\
      Q_{6}
    \end{pmatrix}
  \end{equation}
i les condicions de contorn que imposem són:
\begin{itemize}
  \item Condicions de contorn naturals: $Q_{2} = Q_{3} = Q_{4} = Q_{5} =
    0$, ja que els nodes $2,3,4$ i $5$ són nodes interns i no hi ha
    aplicada cap força puntual. 
  \item Condicions de contorn essencials: $U_{1} = 0$, $U_{6} = 2$.
\end{itemize}
Observem que no podem fixar les variables secundàries $Q_{1}$ i $Q_{6}$,
ja que no coneixem les derivades en les posicions d'aquests nodes.
\end{block}
\end{frame}

\begin{frame}
  \begin{block}{}\justifying
  Aleshores el \emph{sistema reduït} ve donat per les equacions $2,3,4$ i
  $5$ del sistema~\eqref{eq:coupledSystem} passant a la dreta de cada
  igualtat els termes coneguts de la equació, i.e, $K_{i,1} U_{1} + K_{i,6}
  U_{6}$, per $i=2,3,4,5$. D'aquesta manera s'obté,
  \begin{displaymath}
    \begin{split}
      K_{2,2} U_{2} + K_{2,3} U_{3} + K_{2,4} U_{4} + K_{2,5} U_{5} &=
      F_{2} + Q_{2} - K_{2,1} U_{1} - K_{2,6} U_{6}\\ 
      K_{3,2} U_{2} + K_{3,3} U_{3} + K_{3,4} U_{4} + K_{3,5} U_{5} &=
      F_{3} + Q_{3} - K_{3,1} U_{1} - K_{3,6} U_{6}\\ 
      K_{4,2} U_{2} + K_{4,3} U_{3} + K_{4,4} U_{4} + K_{4,5} U_{5} &=
      F_{4} + Q_{4} - K_{4,1} U_{1} - K_{4,6} U_{6}\\ 
      K_{5,2} U_{2} + K_{5,3} U_{3} + K_{5,4} U_{4} + K_{5,5} U_{5} &=
      F_{5} + Q_{5} - K_{5,1} U_{1} - K_{5,6} U_{6} 
  \end{split}
  \end{displaymath}
  o, escrit en forma matricial,
  \begin{equation}\label{eq:reducedSystemM}
    \begin{pmatrix}
      K_{2,2} & K_{2,3} & K_{2,4} & K_{2,5}\\
      K_{3,2} & K_{3,3} & K_{3,4} & K_{3,5}\\
      K_{4,2} & K_{4,3} & K_{4,4} & K_{4,5}\\
      K_{5,2} & K_{5,3} & K_{5,4} & K_{5,5}
    \end{pmatrix}
    \begin{pmatrix}
      U_{2}\\
      U_{3}\\
      U_{4}\\
      U_{5}
    \end{pmatrix} =
    \begin{pmatrix}
      F_{2} \\
      F_{3} \\
      F_{4} \\
      F_{5}
   \end{pmatrix} + 
   \begin{pmatrix}
      Q_{2} \\
      Q_{3} \\
      Q_{4} \\
      Q_{5}
   \end{pmatrix} - 
   \begin{pmatrix}
     K_{2,1} & K_{2,6} \\
     K_{3,1} & K_{3,6} \\
     K_{4,1} & K_{4,6} \\
     K_{5,1} & K_{5,6}
   \end{pmatrix}
   \begin{pmatrix}
     U_{1}\\
     U_{6}
   \end{pmatrix}
  \end{equation}
\end{block}
\end{frame}

\begin{frame}
  \begin{block}{}\justifying
 Aquest és el sistema que resoldrem per a obtenir $U_{2}$, $U_{3}$, $U_{4}$
 i $U_{5}$. Un cop conegudes totes les $U$'s, podem aïllar $Q_{1}$ i
 $Q_{6}$ de la $1^{\text{a}}$ i $6^{\text{a}}$ equació respectivament del
 sistema complet~\eqref{eq:coupledSystem}.
 \end{block}

 \begin{block}{}\justifying
Per últim, si \texttt{fixedNods = [1,6]} és la llista dela nodes ``fixos''
(és a dir els nodes on s'imposen les condicions de contorn essencials) i
\texttt{freeNods = [2,3,4,5]} són tota la resta de nodes,
de~\eqref{eq:reducedSystemM} es veu clarament que el terme
independent (\texttt{Qm}) i la matriu del sistema reduït (\texttt{Km})
s'escriuen en Matlab com
\begin{displaymath}
\begin{array}{l}
  \texttt{Qm = F(freeNods) + Q(freeNods) -
  K(freeNods,fixedNods)*u(fixedNods);}\\
  \texttt{Km = K(freeNods,freeNods);}
\end{array}
\end{displaymath}
que són les comandes que apareixen als guions de les pràctiques.
\end{block}
\end{frame}
\end{document}

\begin{frame}
   % Print the title page as the first slide
    \titlepage
\end{frame}

\begin{frame}{Overview}
    % Throughout your presentation, if you choose to use \section{} and \subsection{} commands, these will automatically be printed on this slide as an overview of your presentation
    \tableofcontents
\end{frame}

%------------------------------------------------
\section{First Section}
%------------------------------------------------

\begin{frame}{Bullet Points}
    \begin{itemize}
        \item Lorem ipsum dolor sit amet, consectetur adipiscing elit
        \item Aliquam blandit faucibus nisi, sit amet dapibus enim tempus eu
        \item Nulla commodo, erat quis gravida posuere, elit lacus lobortis est, quis porttitor odio mauris at libero
        \item Nam cursus est eget velit posuere pellentesque
        \item Vestibulum faucibus velit a augue condimentum quis convallis nulla gravida
    \end{itemize}
\end{frame}

%------------------------------------------------

\begin{frame}{Blocks of Highlighted Text}
    In this slide, some important text will be \alert{highlighted} because it's important. Please, don't abuse it.

    \begin{block}{Block}
        Sample text
    \end{block}

    \begin{alertblock}{Alertblock}
        Sample text in red box
    \end{alertblock}

    \begin{examples}
        Sample text in green box. The title of the block is ``Examples".
    \end{examples}
\end{frame}

%------------------------------------------------

\begin{frame}{Multiple Columns}
    \begin{columns}[c] % The "c" option specifies centered vertical alignment while the "t" option is used for top vertical alignment

        \column{.45\textwidth} % Left column and width
        \textbf{Heading}
        \begin{enumerate}
            \item Statement
            \item Explanation
            \item Example
        \end{enumerate}

        \column{.5\textwidth} % Right column and width
        Lorem ipsum dolor sit amet, consectetur adipiscing elit. Integer lectus nisl, ultricies in feugiat rutrum, porttitor sit amet augue. Aliquam ut tortor mauris. Sed volutpat ante purus, quis accumsan dolor.

    \end{columns}
\end{frame}

%------------------------------------------------
\section{Second Section}
%------------------------------------------------

\begin{frame}{Table}
    \begin{table}
        \begin{tabular}{l l l}
            \toprule
            \textbf{Treatments} & \textbf{Response 1} & \textbf{Response 2} \\
            \midrule
            Treatment 1         & 0.0003262           & 0.562               \\
            Treatment 2         & 0.0015681           & 0.910               \\
            Treatment 3         & 0.0009271           & 0.296               \\
            \bottomrule
        \end{tabular}
        \caption{Table caption}
    \end{table}
\end{frame}

%------------------------------------------------

\begin{frame}{Theorem}
    \begin{theorem}[Mass--energy equivalence]
        $E = mc^2$
    \end{theorem}
\end{frame}

%------------------------------------------------

\begin{frame}{Figure}
    Uncomment the code on this slide to include your own image from the same directory as the template .TeX file.
    %\begin{figure}
    %\includegraphics[width=0.8\linewidth]{test}
    %\end{figure}
\end{frame}

%------------------------------------------------

\begin{frame}[fragile] % Need to use the fragile option when verbatim is used in the slide
    \frametitle{Citation}
    An example of the \verb|\cite| command to cite within the presentation:\\~

    This statement requires citation \cite{p1}.
\end{frame}

%------------------------------------------------

\begin{frame}{References}
    % Beamer does not support BibTeX so references must be inserted manually as below
    \footnotesize{
        \begin{thebibliography}{99}
            \bibitem[Smith, 2012]{p1} John Smith (2012)
            \newblock Title of the publication
            \newblock \emph{Journal Name} 12(3), 45 -- 678.
        \end{thebibliography}
    }
\end{frame}

%------------------------------------------------

\begin{frame}
    \Huge{\centerline{\textbf{The End}}}
\end{frame}

%----------------------------------------------------------------------------------------

\end{document}
